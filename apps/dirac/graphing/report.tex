\documentclass[10pt]{article}
\usepackage{bm, url, graphicx, amsmath, xcolor}
\nonstopmode

\title{DiRAC Benchmarks}
\author{...}

\begin{document}

\maketitle

\section{DiAL: Data Intensive at Leicester}

\subsection{Overview{\footnote{see \url{https://www2.le.ac.uk/offices/itservices/ithelp/services/hpc/dirac/about}}}
}

The University of Leicester hosts one of five High Performance Computing (HPC) systems that forms
part of the DiRAC service. The focus of the service is to support data intensive science problems
in Theoretical Astrophysics and Particle Physics. The DIaL service provides:

\begin{itemize}
	\item 400 dual-socket nodes with 36 2.3 GHz intel skylake cores, 192 GB RAM per node.
	\item 3 large memory nodes: 36 skylake cores, 1.5TB RAM per node.
	\item 1 large memory node with 144 skylake 3GHz cores, 6 TB RAM, supporting larger shared memory codes.
	\item EDR Infiniband Interconnect (2:1 blocking)
	\item 3.3 TB of Lustre parallel storage, supporting over 40GB/s read/write.
	\item 100 GB NVMEoFC storage - experimental data intensive file store with a theoretical peak read/write of 50GB/s.
	\item All nodes run linux CentOS 7 or RHEL 7
\end{itemize}

\clearpage
\subsection{Benchmarks}

{\color{orange} WARNING:} Scaling benchmarks are plotted relative to the lowest value of the
performance metric and are \emph{not} comparable across systems.

\subsubsection{Ramses}
Ramses is a program to model astrophysical systems, featuring
self-gravitating, magnetised, compressible, radiative fluid flows. It is based
on the Adaptive Mesh Refinement (AMR) technique on a fully-threaded graded octree.\footnote{see \url{https://bitbucket.org/rteyssie/ramses/src/master/}}

{\color{red} Some outline of the specific system(s) used for the scaling benchmarks}

\begin{figure*}[h!]
	\centering
	\includegraphics[height=5cm]{Ramses_strong.pdf}
	\vspace{0.1cm}
	\hrule
	\vspace{0.01cm}
	\caption{Ramses strong scaling benchmark.}
	\label{fig::ramses_ss}
\end{figure*}


\begin{figure*}[h!]
	\centering
	\includegraphics[height=5cm]{Ramses_weak.pdf}
	\vspace{0.1cm}
	\hrule
	\vspace{0.01cm}
	\caption{Ramses weak scaling benchmark.}
	\label{fig::ramses_ws}
\end{figure*}


\clearpage
\subsubsection{Sphng}
Smoothed-particle hydrodynamics code used to simulate the mechanics of
continuum media.

{\color{red} Some outline of the specific system(s) used for the scaling benchmarks}


\begin{figure*}[h!]
	\centering
	\includegraphics[height=5cm]{Sphng_strong.pdf}
	\vspace{0.1cm}
	\hrule
	\vspace{0.01cm}
	\caption{Sphng strong scaling benchmark.}
	\label{fig::sphng_ss}
\end{figure*}


\begin{figure*}[h!]
	\centering
	\includegraphics[height=5cm]{Sphng_weak.pdf}
	\vspace{0.1cm}
	\hrule
	\vspace{0.01cm}
	\caption{Sphng weak scaling benchmark.}
	\label{fig::sphng_ws}
\end{figure*}


\clearpage
\subsubsection{Trove}
TROVE (Theoretical ROtational Vibrational Energies) is a suite of programs
primarily designed for the calculation of molecular infrared line lists.\footnote{see \url{https://github.com/Trovemaster/TROVE}}


\begin{figure*}[h!]
	\centering
	\includegraphics[height=5cm]{trove_strong_12N.pdf}
	\vspace{0.1cm}
	\hrule
	\vspace{0.01cm}
	\caption{TROVE strong scaling benchmark.}
	\label{fig::trove_ss_12N}
\end{figure*}

\begin{figure*}[h!]
	\centering
	\includegraphics[height=5cm]{trove_strong_16N.pdf}
	\vspace{0.1cm}
	\hrule
	\vspace{0.01cm}
	\caption{TROVE strong scaling benchmark.}
	\label{fig::trove_ss_16N}
\end{figure*}


\clearpage
\subsection{Summary}

{\color{red} TODO}



\end{document}